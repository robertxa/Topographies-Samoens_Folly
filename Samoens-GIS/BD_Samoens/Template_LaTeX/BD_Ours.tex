\documentclass[sommairechap,stylexrobert]{BD_gix}
 
 \usepackage{multirow}
 \usepackage{eso-pic}
 \usepackage{everyshi}
 \usepackage{ifthen,calc}
 \usepackage{pdfpages}
  \usepackage{natbib}
 
 
\begin{document}
% ==================================================================
 
\titleFR{Inventaire sp�l�ologique du massif du Folly, Syst�me de la grotte aux Ours (Samo�ns, 74)}

% Auteur : 
	\author{Groupe Sp�l�ologique Vulcain}
% Adresse email
	\address{contact@groupe-speleo-vulcain.com}
% Adresse Postale
	\lieu{G.S. Vulcain - 26, rue Sidoine Apollinaire - 69009 Lyon - France}
 
% ==================================================================
% DEDICACE
\dedicate{ }
 
% ==================================================================
% DEBUT DE LA PREFACE
\beforepreface

%\cleardoublepage
 
% table des mati�res g�n�rale
%\large{
\renewcommand{\baselinestretch}{1.2} % interligne
{\large\tableofcontents}
%}
% ==================================================================
% Gestion de la mise en page de tous les chapitres qui suivent :
\afterpreface
%\renewcommand{\baselinestretch}{0.9} % interligne
 
% ==================================================================
% AVANT-PROPOS
%\include{intro}
%\includepdf[landscape]{Images/Inventaire_samoens_2010_3.pdf}
\includepdf[angle=90]{Images/Inventaire_samoens_2010_3.pdf}
\includepdf[landscape]{Images/Inventaire_samoens_2010_4.pdf}
%\adjustmtc
 
% ==================================================================
% CONTENU GENERAL

\chapter{Syst�me de la grotte aux Ours}

\begin{chapintro}
  \malettrine{Z}{ones} abord�es : 
  
	\vspace{3cm}
  
	  \begin{center}
	 \begin{tabular}{ll}
		Combe du A21 & \textbf{A} \\
		Plan du Velar & \textbf{PV} \\
	 \end{tabular}
	 
	\vspace{3cm}
	------------------------------------------------
  	\end{center}
\end{chapintro}


%%%%%%%%%%%%%%%%%%%%%%%%%%%%%%%%%%%%%%%%%%%%%%%%%%%%%%%%%%%%%%%%%%%%%%%%%%%%%%%%%%%%%%%%%%%%%%%%%%%%%%

% Ici, on peut rajouter des pages blanches avec la commande :
% \newpage

% On peut aussi rajouter les images d'intro de la section si besoin :
%\begin{figure}[ht]
%\begin{center}
%	\includegraphics[width=0.95\linewidth]{Images/ermoy-reseauCP.png}
%	\includegraphics[[width=0.95\linewidth, angle=90]{Images/ermoy-reseauCP.png} %l'image est tourn�e de 90�
%	\caption{\it{Texte L�gende si besoin}}
%  \label{label} % Pour faire un appel dans le text avec la commande \ref{label} si besoin
%\end{center}
%\end{figure}

% Insert tableau
% Page Blanche si besoin
\clearpage

{\scriptsize
\begin{center}
\setlength{\extrarowheight}{2pt}
\setlongtables

\begin{longtable} {|m{1cm} |m{0.8cm} |m{0.8cm} |m{1.8cm} |m{7cm} |m{3cm}|}
\hline \hline Nom & Prof. (m) & Dev. (m) & Date & Remarque & R�f�rences \\ \hline \hline \endhead
 \hline \hline \endfoot

	\rowcolor{lightgray} 
	BA1 & -1 & 1 & ??? & Puits �rod� dans la falaise des Barmes & ??? \\\hline

	BA2 & 2 & 10 & 30/05/09 & Galerie � d�sober & E.V. 67   \\\hline

	\rowcolor{lightgray} 
	CP1 & -60 & 80 & 01/06/74 & Glaci�re \/ Courant d'air � travers blocs & E.V. 16, 33, 34, 42*, 44, 47, 51, 53, 64 \\\hline

	CP2 & -51 & 51 & 05/08/74 & Courant d'air � travers �boulis & E.V. 31, 36, 58* \\\hline

\end{longtable}  
\end{center}
}

% Page Blanche si besoin
%\cleardoublepage

\newpage


%%%%%%%%%%%%%%%%%%%%%%%%%%%%%%%%%%%%%%%%%%%%%%%%%%%%%%%%%%%%%%%%%%%%%%%%%%%%%%%%%%%%%%%%%%%%%%%%%%%%%%
\section{Combe du A21}

%\resume{blablabla}
% Rajout des images d'intro de la section si besoin
%\begin{figure}[ht]
%\begin{center}
%	\includegraphics[width=0.85\linewidth]{Images/ermoy-reseauCP.png}
%	\includegraphics[[width=0.95\linewidth, angle=90]{Images/ermoy-reseauCP.png} %l'image est tourn�e de 90�
%	\caption*{\it{Texte L�gende si besoin}}
%  \label{label} % Pour faire un appel dans le text avec la commande \ref{label} si besoin
%\end{center}
%\end{figure}

% Commande pour forcer toutes les images � �tre mises ici
%\FloatBarrier

% Appel � la fiche pour chaque trou avec la commande
%\input{Zones/Ours/A21}

% Page blanche si besoin
\cleardoublepage


%%%%%%%%%%%%%%%%%%%%%%%%%%%%%%%%%%%%%%%%%%%%%%%%%%%%%%%%%%%%%%%%%%%%%%%%%%%%%%%%%%%%%%%%%%%%%%%%%%%%%%
\section{Plan du Velar : PV}

%\resume{blablabla}
% Rajout des images d'intro de la section si besoin
%\begin{figure}[ht]
%\begin{center}
%	\includegraphics[width=0.85\linewidth]{Images/ermoy-reseauCP.png}
%	\includegraphics[[width=0.95\linewidth, angle=90]{Images/ermoy-reseauCP.png} %l'image est tourn�e de 90�
%	\caption*{\it{Texte L�gende si besoin}}
%  \label{label} % Pour faire un appel dans le text avec la commande \ref{label} si besoin
%\end{center}
%\end{figure}

% Commande pour forcer toutes les images � �tre mises ici
%\FloatBarrier

% Appel � la fiche pour chaque trou avec la commande
%\input{Zones/PV/PV1}

% Page blanche si besoin
\cleardoublepage


 
% ==================================================================
% ANNEXES
%\appendix
%\include{annexe} 
 
% ==================================================================
% COLOPHON
%\colophon{Ce document a été préparé à l'aide de l'éditeur de texte GNU
%  Emacs et du logiciel de composition typographique \LaTeXe
%  Rajouter comment il est produit ???? et par qui ????.}
 
\end{document}
%%% Local Variables:
%%% mode: latex
%%% TeX-master: t
%%% End:
