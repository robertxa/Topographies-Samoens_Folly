\documentclass[sommairechap,stylexrobert]{BD_gix}	% D�finition du style
 
 % On appelle quelques packages untiles
 \usepackage{multirow}
 \usepackage{eso-pic}
 \usepackage{everyshi}
 \usepackage{ifthen,calc}
 \usepackage{pdfpages}
  \usepackage{natbib}
 
 
\begin{document}	% Il faut toujours cette ligne pour d�marrer un document
% ==================================================================
 
\titleFR{Inventaire sp�l�ologique du massif du Folly (Samo�ns, 74)}

% Auteur : 
	\author{Groupe Sp�l�ologique Vulcain}
% Adresse email
	\address{contact@groupe-speleo-vulcain.com}
% Adresse Postale
	\lieu{G.S. Vulcain - 26, rue Sidoine Apollinaire - 69009 Lyon - France}
  
% ==================================================================
% DEBUT DE LA PREFACE
\beforepreface

%\cleardoublepage
 
% Table des mati�res g�n�rale
%\large{
\renewcommand{\baselinestretch}{1.2} % interligne
{\large\tableofcontents}
%}
% ==================================================================
% Gestion de la mise en page de tous les chapitres qui suivent :
\afterpreface
%\renewcommand{\baselinestretch}{0.9} % interligne
 
% ==================================================================
% AVANT-PROPOS
%\include{intro}
%\includepdf[landscape]{Images/Inventaire_samoens_2010_3.pdf}
\includepdf[angle=90]{Images/Inventaire_samoens_2010_3.pdf}
\includepdf[landscape]{Images/Inventaire_samoens_2010_4.pdf}
%\adjustmtc
 
% ==================================================================
% CONTENU GENERAL
% On appelle ici les fichiers tex correspondant � chaque zone
\chapter{Syst�me de la Combe aux Puaires}

\begin{chapintro}
 
  \malettrine{Z}{ones} abord�es : 
  
	\vspace{3cm}
  
	  \begin{center}
	 \begin{tabular}{ll}
		Combe aux Puaires & \textbf{CP} \\
		Lapiaz c�t�s Puaires & \textbf{LP} \\
		Pointe Droite & \textbf{DR} \\
		Zone des Lapiez & \textbf{L} \\
		Zone des Lapiez Sup�rieurs & \textbf{LS} \\
		Zone de la t�te � l'Homme & \textbf{T} \\
	 \end{tabular}
	
	\vspace{3cm}
	------------------------------------------------
  	\end{center}
\end{chapintro}


%%%%%%%%%%%%%%%%%%%%%%%%%%%%%%%%%%%%%%%%%%%%%%%%%%%%%%%%%%%%%%%%%%%%%%%%%%%%%%%%%%%%%%%%%%%%%%%%%%%%%%

% Ici, on peut rajouter des pages blanches avec la commande :
% \newpage

% On peut aussi rajouter les images d'intro de la section si besoin :
%\begin{figure}[ht]
%\begin{center}
%	\includegraphics[width=0.95\linewidth]{Images/ermoy-reseauCP.png}
%	\includegraphics[[width=0.95\linewidth, angle=90]{Images/ermoy-reseauCP.png} %l'image est tourn�e de 90�
%	\caption{\it{Texte L�gende si besoin}} % Pas obligatoire
%  \label{label} % Pour faire un appel dans le text avec la commande \ref{label} si besoin ; Pas obligatoire
%\end{center}
%\end{figure}

% Pour ins�rer une page pdf en pleine page :
%\includepdf[landscape]{Images/Inventaire_samoens_2010_3.pdf}	% Mode paysage
%\includepdf[angle=90]{Images/Inventaire_samoens_2010_3.pdf}	% Mode portrait avec rotation de 90�
%\includepdf[landscape]{Images/Inventaire_samoens_2010_3.pdf}	% Mode portrait


% appel au tableau
% Page Blanche si besoin
\clearpage

{\scriptsize	% On change la casse pour du tout petit
\begin{center}	% On centre le tableau
\setlength{\extrarowheight}{2pt}
\setlongtables

\begin{longtable} {|m{1cm} |m{0.8cm} |m{0.8cm} |m{1.8cm} |m{7cm} |m{3cm}|}	% On ouvre le tableau et on d�finit les colonnes, leur taille et leur lignes verticales
\hline \hline textbf{Nom} & textbf{Prof. (m)} & textbf{Dev. (m)} & textbf{Date} & textbf{Remarques} & textbf{R�f�rences} \\ \hline \hline \endhead	% On d�finit le header
 \hline \hline \endfoot
% Dans un tableau, on s�pare chaque colonne avec &, et on revient � la ligne avec \\. \hline, c'est pour g�n�rer une s�paration horizontale sous forme de ligne
	\rowcolor{lightgray} % Ca permet de griser la ligne en cours
	BA1 & -1 & 1 & ??? & Puits �rod� dans la falaise des Barmes & ??? \\\hline

	BA2 & 2 & 10 & 30/05/09 & Galerie � d�sober & E.V. 67   \\\hline

	\rowcolor{lightgray} 
	CP1 & -60 & 80 & 01/06/74 & Glaci�re \/ Courant d'air � travers blocs & E.V. 16, 33, 34, 42*, 44, 47, 51, 53, 64 \\\hline

	CP2 & -51 & 51 & 05/08/74 & Courant d'air � travers �boulis & E.V. 31, 36, 58* \\\hline

\end{longtable}  
\end{center}
} % Fermer le bloc scriptsize

% Page Blanche si besoin
%\cleardoublepage

\newpage

%%%%%%%%%%%%%%%%%%%%%%%%%%%%%%%%%%%%%%%%%%%%%%%%%%%%%%%%%%%%%%%%%%%%%%%%%%%%%%%%%%%%%%%%%%%%%%%%%%%%%%
\section{Zone de la Combe aux Puaires : CP}

% Rajout des images d'intro de la section si besoin
\begin{figure}[ht]
\begin{center}
	\includegraphics[width=0.85\linewidth]{Images/ermoy-reseauCP.png}
\end{center}
\end{figure}

% Commande pour forcer toutes les images � �tre mises ici
\FloatBarrier

% Appel � la fiche pour chaque trou avec la commande
% Si nous voulons que ce soit sur une nouvelle page
\newpage

%%%%%%%%%%%%%%%%%%%%%%%%%%%%%%%%%%%%%%%%%%%%%%%%%%%%%%%%%%%%%%%%%%%%%%%%%%%%%%%%%%%%%%%%%%%%%%%%%%%%%%
% D�finition des variables
\trou{CP1} % Nom(s) du trou

\systeme{Combe aux Puaires}	% Syst�me auquel il appartient

\developpement{80} % sans unit�
\profondeur{-60} % sans unit�

%\lat{46,113547} % sans unit�
\lat{} % sans unit�
\longi{6,790496} % sans unit�
\altitude{1943} % sans unit�
\XUTM{329265} % sans unit�
\YUTM{5109037} % sans unit�

\situation{
Il s'ouvre 100 m en contrebas du sentier qui serpente le long de la combe aux Puaires.
}

\decouverte{
01/06/74
}

\histoire{
Les premi�res visites � ce gouffre ont commenc� d�s 1974 et 1976 (E.V. 33) En septembre 1982, P. Bergeron, J.B. de Chabalier et C. Ohl l'ont revu et topographi�. \\
� \\
� \\
 Le 17/06/06, X. Robert, A. Ohl, A. Mallard, B. et St. Lips refouillent la cavit�. Le boyau englac� connu par Pernard en 1973 mesure maintenant 2 m 50 de haut pour 4 � 5 m de large. Il donne sur une plateforme sous le glacier. L'ancienne suite est bouch�e, et vers le sud, s'ouvrent 2 puits. Une galerie m�ne sous le glacier, � -40 m environ, et une autre galerie basse (1 m 50 de haut maximum) donne sur une petite salle basse, dont le sol est recouvert par 50 cm de glace de regel translucide. Sur un des c�t�s, il y a un boyasson de 15 cm de diam�tre qui souffle un courant d'air puissant : �a siffle. Il ne m'en faut pas plus pour commencer � gratter, pendant que le reste de lՎquipe explore la seconde branche. Au bout de deux heures de travail, le passage n�est toujours pas p�n�trable. TPST : 4 h. \\
 Le 9/07/06, X. Robert, J. Arnaud, Fl. Colinet et P. Plantier entreprenne le dynamitage d'un m�andre s'ouvrant � -40 m environ.
}

\descrip{
Le porche d'entr�e, aux dimensions inhabituelles sur le Folly (6 x 10 m) donne sur un n�v� fortement inclin�. 15 m plus bas, la salle occup�e par un lac de glace de regel, est perc�e par un tunnel creus� dans la glace. C'est l'abondante arriv�e d'eau tombant du plafond qui permet ce passage. Il faut se glisser ensuite dans un boyau englac� pour d�boucher sous le n�v�. Dix m�tres plus bas, il faut fractionner sur la paroi oppos�e au moment o� nous quittons la glace. De l�, deux possibilit�s s'offrent � nous : 1) Descendre la verticale (P 15) . On se retrouve alors entre le miroir de faille et le n�v� ( l= 1 � 0,4 m). Le passage se r�tr�cit de plus en plus � mesure que l'on descend. Le fond est colmat� par des blocs dans un pincement de la faille. On peut faire le tour du culot de glace (5 m) et m�me se glisser sous celui-ci. Aucune possibilit� de continuation n'est visible (pas de courant d'air). Profondeur -60 m. 2) On rejoint le plan inclin� du glacier que l'on descend jusqu'� une salle colmat�e par des blocs. Un violent courant d'air souffle d'un boyau que nous avons commenc� � d�sobstruer mais il faudrait utiliser des explosifs. Ce sera un de nos prochains objectifs,
}

\rem{
Le porche d'entr�e du CP1 s'ouvre dans le S�nonien �pais de 20 m � cet endroit La traditionnelle couche d'Albien est ici lamin�e et c�est dans l'Urgonien que se d�veloppe le reste de la cavit�. Le CP01 est creus� aux d�pens de deux failles de compression importantes : une perpendiculaire et l�autre parall�le � l'axe du synclinal, Les deux pr�sentent un rejet important C'est le gouffre-glacier plus important que nous connaissions � l�heure actuelle sur le massif. La glace y est stratifi�e horizontalement et atteint une �paisseur de 60 m, puisque nous le suivons jusqu'� l'�boulis terminal.
}

\biblio{
E.V. 16, 33, 34, 42*, 44, 47, 51, 53, 64
}

%%%%%%%%%%%%%%%%%%%%%%%%%%%%%%%%%%%%%%%%%%%%%%%%%%%%%%%%%%%%%%%%%%%%%%%%%%%%%%%%%%%%%%%%%%%%%%%%%%%%%%
% Construction la fiche
\fichetrougd

% rajout des figures (jpg ou png)
\begin{figure}[ht]
	\begin{center}
		\includegraphics[width=0.95\linewidth]{Zones/CP/topos/CP1-plan2.png}
	\end{center}
\end{figure}
%\begin{figure}[ht]
%	\begin{center}
%		\includegraphics[width=0.95\linewidth]{Zones/CP/topos/CP1-plan.png}
%	\end{center}
%\end{figure}
%\begin{figure}[ht]
%	\begin{center}
%		\includegraphics[width=0.95\linewidth]{Zones/CP/topos/CP1-coupe.png}
%	\end{center}
%\end{figure}
% impose l'impression des images ici
\FloatBarrier

%\newpage
%\cleardoublepage

%\newpage

%%%%%%%%%%%%%%%%%%%%%%%%%%%%%%%%%%%%%%%%%%%%%%%%%%%%%%%%%%%%%%%%%%%%%%%%%%%%%%%%%%%%%%%%%%%%%%%%%%%%%%
% D�finition des variables
\trou{CP2} % Nom(s) du trou

\systeme{Combe aux Puaires}	% Syst�me auquel il appartient

\developpement{51} % sans unit�
\profondeur{-51} % sans unit�

\lat{46,116134} % sans unit�
\longi{6,797549} % sans unit�
\altitude{} % sans unit�
\XUTM{329818} % sans unit�
\YUTM{5109309} % sans unit�

\situation{
entr�e de la combe aux puaires. (en bas des "grand trous")
}

\decouverte{
05/08/74
}

\histoire{
Explor� le 05/08/74 par P. Cahinght et Serge, Topo effectu�e (?). revu par Christian Rigaldie et Phillippe Lavabre en aout 77. \\
Revu le 12/07/00 par P. Cahingt et par S. Lips.
}

\descrip{
Puits de 51 m�tres dans une diaclase importante, il a une forme ovale caract�ristique. Le courant d'air se perd dans la salle terminale (4X10 m) � travers un �boulis instable et infranchissable.
}

\rem{
termin�, courant d'air
}

\biblio{
E.V. 31, 36, 58*
}

%%%%%%%%%%%%%%%%%%%%%%%%%%%%%%%%%%%%%%%%%%%%%%%%%%%%%%%%%%%%%%%%%%%%%%%%%%%%%%%%%%%%%%%%%%%%%%%%%%%%%%
% Construction la fiche

% rajout des figures (jpg ou png)
%\begin{wrapfigure}[20]{r}{6cm}
%	\begin{center}
%		\fbox{\includegraphics[width=1\linewidth]{Zones/CP/topos/CP02.png}}
%	\end{center}
%\end{wrapfigure}
%\fichetrou

\topopt{Zones/CP/topos/CP02.png}
\fichetroupt

% impose l'impression des images ici
\FloatBarrier

%\newpage
%\cleardoublepage
 %\newpage

%%%%%%%%%%%%%%%%%%%%%%%%%%%%%%%%%%%%%%%%%%%%%%%%%%%%%%%%%%%%%%%%%%%%%%%%%%%%%%%%%%%%%%%%%%%%%%%%%%%%%%
% D�finition des variables
\trou{CP4} % Nom(s) du trou

\systeme{Combe aux Puaires}	% Syst�me auquel il appartient

\developpement{22} % sans unit�
\profondeur{-22} % sans unit�

\lat{46,118052} % sans unit�
\longi{6,799403} % sans unit�
\altitude{2197} % sans unit�
\XUTM{329967} % sans unit�
\YUTM{5109518} % sans unit�

\situation{
sur les lapiaz bordant la combe aux puaires, sur la m�me faille que le CP05 et le CP06.
}

\decouverte{
07/08/74
}

\histoire{
explor� le 07/08/74 par Rigaldie C.
}

\descrip{
puits de 22 m sur faille obstru� par �boulis et assez �troit. Sur le cot�, � trois m�tres du fond, la faille se prolonge lat�ralement mais reste trop �troite.
}

\rem{}

\biblio{
E.V. 31, 36*
}

%%%%%%%%%%%%%%%%%%%%%%%%%%%%%%%%%%%%%%%%%%%%%%%%%%%%%%%%%%%%%%%%%%%%%%%%%%%%%%%%%%%%%%%%%%%%%%%%%%%%%%
% Construction la fiche

\topopt{Zones/CP/topos/CP4.png}
\fichetroupt

% impose l'impression des images ici
\FloatBarrier

%\newpage
%\cleardoublepage

% Si nous voulons que ce soit sur une nouvelle page
\newpage

%%%%%%%%%%%%%%%%%%%%%%%%%%%%%%%%%%%%%%%%%%%%%%%%%%%%%%%%%%%%%%%%%%%%%%%%%%%%%%%%%%%%%%%%%%%%%%%%%%%%%%
% D�finition des variables
\trou{CP1} % Nom(s) du trou

\systeme{Combe aux Puaires}	% Syst�me auquel il appartient

\developpement{80} % sans unit�
\profondeur{-60} % sans unit�

%\lat{46,113547} % sans unit�
\lat{} % sans unit�
\longi{6,790496} % sans unit�
\altitude{1943} % sans unit�
\XUTM{329265} % sans unit�
\YUTM{5109037} % sans unit�

\situation{
Il s'ouvre 100 m en contrebas du sentier qui serpente le long de la combe aux Puaires.
}

\decouverte{
01/06/74
}

\histoire{
Les premi�res visites � ce gouffre ont commenc� d�s 1974 et 1976 (E.V. 33) En septembre 1982, P. Bergeron, J.B. de Chabalier et C. Ohl l'ont revu et topographi�. \\
� \\
� \\
 Le 17/06/06, X. Robert, A. Ohl, A. Mallard, B. et St. Lips refouillent la cavit�. Le boyau englac� connu par Pernard en 1973 mesure maintenant 2 m 50 de haut pour 4 � 5 m de large. Il donne sur une plateforme sous le glacier. L'ancienne suite est bouch�e, et vers le sud, s'ouvrent 2 puits. Une galerie m�ne sous le glacier, � -40 m environ, et une autre galerie basse (1 m 50 de haut maximum) donne sur une petite salle basse, dont le sol est recouvert par 50 cm de glace de regel translucide. Sur un des c�t�s, il y a un boyasson de 15 cm de diam�tre qui souffle un courant d'air puissant : �a siffle. Il ne m'en faut pas plus pour commencer � gratter, pendant que le reste de lՎquipe explore la seconde branche. Au bout de deux heures de travail, le passage n�est toujours pas p�n�trable. TPST : 4 h. \\
 Le 9/07/06, X. Robert, J. Arnaud, Fl. Colinet et P. Plantier entreprenne le dynamitage d'un m�andre s'ouvrant � -40 m environ.
}

\descrip{
Le porche d'entr�e, aux dimensions inhabituelles sur le Folly (6 x 10 m) donne sur un n�v� fortement inclin�. 15 m plus bas, la salle occup�e par un lac de glace de regel, est perc�e par un tunnel creus� dans la glace. C'est l'abondante arriv�e d'eau tombant du plafond qui permet ce passage. Il faut se glisser ensuite dans un boyau englac� pour d�boucher sous le n�v�. Dix m�tres plus bas, il faut fractionner sur la paroi oppos�e au moment o� nous quittons la glace. De l�, deux possibilit�s s'offrent � nous : 1) Descendre la verticale (P 15) . On se retrouve alors entre le miroir de faille et le n�v� ( l= 1 � 0,4 m). Le passage se r�tr�cit de plus en plus � mesure que l'on descend. Le fond est colmat� par des blocs dans un pincement de la faille. On peut faire le tour du culot de glace (5 m) et m�me se glisser sous celui-ci. Aucune possibilit� de continuation n'est visible (pas de courant d'air). Profondeur -60 m. 2) On rejoint le plan inclin� du glacier que l'on descend jusqu'� une salle colmat�e par des blocs. Un violent courant d'air souffle d'un boyau que nous avons commenc� � d�sobstruer mais il faudrait utiliser des explosifs. Ce sera un de nos prochains objectifs,
}

\rem{
Le porche d'entr�e du CP1 s'ouvre dans le S�nonien �pais de 20 m � cet endroit La traditionnelle couche d'Albien est ici lamin�e et c�est dans l'Urgonien que se d�veloppe le reste de la cavit�. Le CP01 est creus� aux d�pens de deux failles de compression importantes : une perpendiculaire et l�autre parall�le � l'axe du synclinal, Les deux pr�sentent un rejet important C'est le gouffre-glacier plus important que nous connaissions � l�heure actuelle sur le massif. La glace y est stratifi�e horizontalement et atteint une �paisseur de 60 m, puisque nous le suivons jusqu'� l'�boulis terminal.
}

\biblio{
E.V. 16, 33, 34, 42*, 44, 47, 51, 53, 64
}

%%%%%%%%%%%%%%%%%%%%%%%%%%%%%%%%%%%%%%%%%%%%%%%%%%%%%%%%%%%%%%%%%%%%%%%%%%%%%%%%%%%%%%%%%%%%%%%%%%%%%%
% Construction la fiche
\fichetrougd

% rajout des figures (jpg ou png)
\begin{figure}[ht]
	\begin{center}
		\includegraphics[width=0.95\linewidth]{Zones/CP/topos/CP1-plan2.png}
	\end{center}
\end{figure}
%\begin{figure}[ht]
%	\begin{center}
%		\includegraphics[width=0.95\linewidth]{Zones/CP/topos/CP1-plan.png}
%	\end{center}
%\end{figure}
%\begin{figure}[ht]
%	\begin{center}
%		\includegraphics[width=0.95\linewidth]{Zones/CP/topos/CP1-coupe.png}
%	\end{center}
%\end{figure}
% impose l'impression des images ici
\FloatBarrier

%\newpage
%\cleardoublepage


% Page Blanche si besoin
\cleardoublepage


%%%%%%%%%%%%%%%%%%%%%%%%%%%%%%%%%%%%%%%%%%%%%%%%%%%%%%%%%%%%%%%%%%%%%%%%%%%%%%%%%%%%%%%%%%%%%%%%%%%%%%
\section{Lapiaz c�t� Puaires : LP}

% Rajout des images d'intro de la section si besoin
% Commande pour forcer toutes les images � �tre mises ici
%\FloatBarrier

% Appel � la fiche pour chaque trou avec la commande
%\input{LP/LP1}

% Page Blanche si besoin
\cleardoublepage


%%%%%%%%%%%%%%%%%%%%%%%%%%%%%%%%%%%%%%%%%%%%%%%%%%%%%%%%%%%%%%%%%%%%%%%%%%%%%%%%%%%%%%%%%%%%%%%%%%%%%%
\section{Pointe Droite : DR}

% Rajout des images d'intro de la section si besoin
% Commande pour forcer toutes les images � �tre mises ici
%\FloatBarrier

% Appel � la fiche pour chaque trou avec la commande
%\input{DR/DR1}

% Page Blanche si besoin
\cleardoublepage


%%%%%%%%%%%%%%%%%%%%%%%%%%%%%%%%%%%%%%%%%%%%%%%%%%%%%%%%%%%%%%%%%%%%%%%%%%%%%%%%%%%%%%%%%%%%%%%%%%%%%%
\section{Zone des Lapiez : L}

% Rajout des images d'intro de la section si besoin
% Commande pour forcer toutes les images � �tre mises ici
%\FloatBarrier

% Appel � la fiche pour chaque trou avec la commande
%\input{L/L1}

% Page Blanche si besoin
\cleardoublepage


%%%%%%%%%%%%%%%%%%%%%%%%%%%%%%%%%%%%%%%%%%%%%%%%%%%%%%%%%%%%%%%%%%%%%%%%%%%%%%%%%%%%%%%%%%%%%%%%%%%%%%
\section{Zone des Lapiez Sup�rieurs : LS}

% Rajout des images d'intro de la section si besoin

% Commande pour forcer toutes les images � �tre mises ici
%\FloatBarrier

% Appel � la fiche pour chaque trou avec la commande
%\ininput{LS/LS1}

% Page Blanche si besoin
\cleardoublepage


%%%%%%%%%%%%%%%%%%%%%%%%%%%%%%%%%%%%%%%%%%%%%%%%%%%%%%%%%%%%%%%%%%%%%%%%%%%%%%%%%%%%%%%%%%%%%%%%%%%%%%
\section{Zone de la T�te � l'Homme : T}

% Rajout des images d'intro de la section si besoin
% Commande pour forcer toutes les images � �tre mises ici
%\FloatBarrier

% Appel � la fiche pour chaque trou avec la commande
%\input{T/T1}

% Page Blanche si besoin
\cleardoublepage

\chapter{Syst�me du Jean-Bernard}

\begin{chapintro}
  \malettrine{Z}{ones} abord�es : 
 
	\vspace{3cm}
  
	  \begin{center} 
	 \begin{tabular}{ll}
		Zone des A & \textbf{A} \\
		Zone des B & \textbf{B} \\
		Zone des C & \textbf{C} \\
		Zone des D & \textbf{D} \\
		Zone des E & \textbf{E} \\
		Zone du vallon des Chambres & \textbf{CH} \\
	 \end{tabular}
	 
	\vspace{3cm}
	------------------------------------------------
  	\end{center}
\end{chapintro}


%%%%%%%%%%%%%%%%%%%%%%%%%%%%%%%%%%%%%%%%%%%%%%%%%%%%%%%%%%%%%%%%%%%%%%%%%%%%%%%%%%%%%%%%%%%%%%%%%%%%%%

% Ici, on peut rajouter des pages blanches avec la commande :
% \newpage

% On peut aussi rajouter les images d'intro de la section si besoin :
%\begin{figure}[ht]
%\begin{center}
%	\includegraphics[width=0.95\linewidth]{Images/ermoy-reseauCP.png}
%	\includegraphics[[width=0.95\linewidth, angle=90]{Images/ermoy-reseauCP.png} %l'image est tourn�e de 90�
%	\caption{\it{Texte L�gende si besoin}}
%  \label{label} % Pour faire un appel dans le text avec la commande \ref{label} si besoin
%\end{center}
%\end{figure}

% Insert tableau
\input{Tableaux/tableauJB}
\newpage

%%%%%%%%%%%%%%%%%%%%%%%%%%%%%%%%%%%%%%%%%%%%%%%%%%%%%%%%%%%%%%%%%%%%%%%%%%%%%%%%%%%%%%%%%%%%%%%%%%%%%%
\section{Zone A}

%\resume{blablabla}
% Rajout des images d'intro de la section si besoin
%\begin{figure}[ht]
%\begin{center}
%	\includegraphics[width=0.85\linewidth]{Images/ermoy-reseauCP.png}
%	\includegraphics[[width=0.95\linewidth, angle=90]{Images/ermoy-reseauCP.png} %l'image est tourn�e de 90�
%	\caption*{\it{Texte L�gende si besoin}}
%  \label{label} % Pour faire un appel dans le text avec la commande \ref{label} si besoin
%\end{center}
%\end{figure}

% Commande pour forcer toutes les images � �tre mises ici
%\FloatBarrier

% Appel � la fiche pour chaque trou avec la commande
%\input{Zones/A/A1}

% Page blanche si besoin
\cleardoublepage


%%%%%%%%%%%%%%%%%%%%%%%%%%%%%%%%%%%%%%%%%%%%%%%%%%%%%%%%%%%%%%%%%%%%%%%%%%%%%%%%%%%%%%%%%%%%%%%%%%%%%%
\section{Zone B}

% Rajout des images d'intro de la section si besoin
% Commande pour forcer toutes les images � �tre mises ici
%\FloatBarrier

% Appel � la fiche pour chaque trou avec la commande
%\include{Zones/B/B1}

% Page blanche si besoin
\cleardoublepage


%%%%%%%%%%%%%%%%%%%%%%%%%%%%%%%%%%%%%%%%%%%%%%%%%%%%%%%%%%%%%%%%%%%%%%%%%%%%%%%%%%%%%%%%%%%%%%%%%%%%%%
\section{Zone C}

% Rajout des images d'intro de la section si besoin
% Commande pour forcer toutes les images � �tre mises ici
%\FloatBarrier

% Appel � la fiche pour chaque trou avec la commande
%\include{Zones/C/C1}

% Page blanche si besoin
\cleardoublepage


%%%%%%%%%%%%%%%%%%%%%%%%%%%%%%%%%%%%%%%%%%%%%%%%%%%%%%%%%%%%%%%%%%%%%%%%%%%%%%%%%%%%%%%%%%%%%%%%%%%%%%
\section{Zone D}

% Rajout des images d'intro de la section si besoin
% Commande pour forcer toutes les images � �tre mises ici
%\FloatBarrier

% Appel � la fiche pour chaque trou avec la commande
%\include{Zones/D/D1}

% Page blanche si besoin
\cleardoublepage


%%%%%%%%%%%%%%%%%%%%%%%%%%%%%%%%%%%%%%%%%%%%%%%%%%%%%%%%%%%%%%%%%%%%%%%%%%%%%%%%%%%%%%%%%%%%%%%%%%%%%%
\section{Zone E}

% Rajout des images d'intro de la section si besoin
% Commande pour forcer toutes les images � �tre mises ici
%\FloatBarrier

% Appel � la fiche pour chaque trou avec la commande
%\include{Zones/E/E1}

% Page blanche si besoin
\cleardoublepage


%%%%%%%%%%%%%%%%%%%%%%%%%%%%%%%%%%%%%%%%%%%%%%%%%%%%%%%%%%%%%%%%%%%%%%%%%%%%%%%%%%%%%%%%%%%%%%%%%%%%%%
\section{Zone CH}

% Rajout des images d'intro de la section si besoin
% Commande pour forcer toutes les images � �tre mises ici
%\FloatBarrier

% Appel � la fiche pour chaque trou avec la commande
%\include{Zones/CH/CH1}

% Page blanche si besoin
\cleardoublepage



\chapter{Syst�me des Avoudrues}

\begin{chapintro}
  \malettrine{Z}{ones} abord�es : 
  
	\vspace{3cm}
  
	  \begin{center}
	 \begin{tabular}{ll}
		Combe aux Puaires & \textbf{AV} \\
	 \end{tabular}
	 
	\vspace{3cm}
	------------------------------------------------
  	\end{center}
\end{chapintro}


%%%%%%%%%%%%%%%%%%%%%%%%%%%%%%%%%%%%%%%%%%%%%%%%%%%%%%%%%%%%%%%%%%%%%%%%%%%%%%%%%%%%%%%%%%%%%%%%%%%%%%

% Ici, on peut rajouter des pages blanches avec la commande :
% \newpage

% On peut aussi rajouter les images d'intro de la section si besoin :
%\begin{figure}[ht]
%\begin{center}
%	\includegraphics[width=0.95\linewidth]{Images/ermoy-reseauCP.png}
%	\includegraphics[[width=0.95\linewidth, angle=90]{Images/ermoy-reseauCP.png} %l'image est tourn�e de 90�
%	\caption{\it{Texte L�gende si besoin}}
%  \label{label} % Pour faire un appel dans le text avec la commande \ref{label} si besoin
%\end{center}
%\end{figure}

% Insert tableau
\input{Tableaux/tableauAV}
\newpage


%%%%%%%%%%%%%%%%%%%%%%%%%%%%%%%%%%%%%%%%%%%%%%%%%%%%%%%%%%%%%%%%%%%%%%%%%%%%%%%%%%%%%%%%%%%%%%%%%%%%%%
\section{Zone des Avoudrues : AV}

%\resume{blablabla}
% Rajout des images d'intro de la section si besoin
%\begin{figure}[ht]
%\begin{center}
%	\includegraphics[width=0.85\linewidth]{Images/ermoy-reseauCP.png}
%	\includegraphics[[width=0.95\linewidth, angle=90]{Images/ermoy-reseauCP.png} %l'image est tourn�e de 90�
%	\caption*{\it{Texte L�gende si besoin}}
%  \label{label} % Pour faire un appel dans le text avec la commande \ref{label} si besoin
%\end{center}
%\end{figure}

% Commande pour forcer toutes les images � �tre mises ici
%\FloatBarrier

% Appel � la fiche pour chaque trou avec la commande
%\input{Zones/AV/AV1} \input{Zones/AV/AV2}
%\input{Zones/AV/AV3} \input{Zones/AV/AV4}
%\input{Zones/AV/AV5} \input{Zones/AV/AV6}
%\input{Zones/AV/AV7} \input{Zones/AV/AV9}
%\input{Zones/AV/AV8}
%\input{Zones/AV/AV10} \input{Zones/AV/AV11}
%\input{Zones/AV/AV12} \input{Zones/AV/AV13}
%\input{Zones/AV/AV14} \input{Zones/AV/AV15}
%\input{Zones/AV/AV16} \input{Zones/AV/AV17}
%\input{Zones/AV/AV18} \input{Zones/AV/AV19}
%\input{Zones/AV/AV20} \input{Zones/AV/AV21} % A v�rifier, mais �a risque de ne pas tenir sur une page.

% Page blanche si besoin
\cleardoublepage


%%%%%%%%%%%%%%%%%%%%%%%%%%%%%%%%%%%%%%%%%%%%%%%%%%%%%%%%%%%%%%%%%%%%%%%%%%%%%%%%%%%%%%%%%%%%%%%%%%%%%%

\chapter{Syst�me de la grotte aux Ours}

\begin{chapintro}
  \malettrine{Z}{ones} abord�es : 
  
	\vspace{3cm}
  
	  \begin{center}
	 \begin{tabular}{ll}
		Combe du A21 & \textbf{A} \\
		Plan du Velar & \textbf{PV} \\
	 \end{tabular}
	 
	\vspace{3cm}
	------------------------------------------------
  	\end{center}
\end{chapintro}


%%%%%%%%%%%%%%%%%%%%%%%%%%%%%%%%%%%%%%%%%%%%%%%%%%%%%%%%%%%%%%%%%%%%%%%%%%%%%%%%%%%%%%%%%%%%%%%%%%%%%%

% Ici, on peut rajouter des pages blanches avec la commande :
% \newpage

% On peut aussi rajouter les images d'intro de la section si besoin :
%\begin{figure}[ht]
%\begin{center}
%	\includegraphics[width=0.95\linewidth]{Images/ermoy-reseauCP.png}
%	\includegraphics[[width=0.95\linewidth, angle=90]{Images/ermoy-reseauCP.png} %l'image est tourn�e de 90�
%	\caption{\it{Texte L�gende si besoin}}
%  \label{label} % Pour faire un appel dans le text avec la commande \ref{label} si besoin
%\end{center}
%\end{figure}

% Insert tableau
% Page Blanche si besoin
\clearpage

{\scriptsize
\begin{center}
\setlength{\extrarowheight}{2pt}
\setlongtables

\begin{longtable} {|m{1cm} |m{0.8cm} |m{0.8cm} |m{1.8cm} |m{7cm} |m{3cm}|}
\hline \hline Nom & Prof. (m) & Dev. (m) & Date & Remarque & R�f�rences \\ \hline \hline \endhead
 \hline \hline \endfoot

	\rowcolor{lightgray} 
	BA1 & -1 & 1 & ??? & Puits �rod� dans la falaise des Barmes & ??? \\\hline

	BA2 & 2 & 10 & 30/05/09 & Galerie � d�sober & E.V. 67   \\\hline

	\rowcolor{lightgray} 
	CP1 & -60 & 80 & 01/06/74 & Glaci�re \/ Courant d'air � travers blocs & E.V. 16, 33, 34, 42*, 44, 47, 51, 53, 64 \\\hline

	CP2 & -51 & 51 & 05/08/74 & Courant d'air � travers �boulis & E.V. 31, 36, 58* \\\hline

\end{longtable}  
\end{center}
}

% Page Blanche si besoin
%\cleardoublepage

\newpage


%%%%%%%%%%%%%%%%%%%%%%%%%%%%%%%%%%%%%%%%%%%%%%%%%%%%%%%%%%%%%%%%%%%%%%%%%%%%%%%%%%%%%%%%%%%%%%%%%%%%%%
\section{Combe du A21}

%\resume{blablabla}
% Rajout des images d'intro de la section si besoin
%\begin{figure}[ht]
%\begin{center}
%	\includegraphics[width=0.85\linewidth]{Images/ermoy-reseauCP.png}
%	\includegraphics[[width=0.95\linewidth, angle=90]{Images/ermoy-reseauCP.png} %l'image est tourn�e de 90�
%	\caption*{\it{Texte L�gende si besoin}}
%  \label{label} % Pour faire un appel dans le text avec la commande \ref{label} si besoin
%\end{center}
%\end{figure}

% Commande pour forcer toutes les images � �tre mises ici
%\FloatBarrier

% Appel � la fiche pour chaque trou avec la commande
%\input{Zones/Ours/A21}

% Page blanche si besoin
\cleardoublepage


%%%%%%%%%%%%%%%%%%%%%%%%%%%%%%%%%%%%%%%%%%%%%%%%%%%%%%%%%%%%%%%%%%%%%%%%%%%%%%%%%%%%%%%%%%%%%%%%%%%%%%
\section{Plan du Velar : PV}

%\resume{blablabla}
% Rajout des images d'intro de la section si besoin
%\begin{figure}[ht]
%\begin{center}
%	\includegraphics[width=0.85\linewidth]{Images/ermoy-reseauCP.png}
%	\includegraphics[[width=0.95\linewidth, angle=90]{Images/ermoy-reseauCP.png} %l'image est tourn�e de 90�
%	\caption*{\it{Texte L�gende si besoin}}
%  \label{label} % Pour faire un appel dans le text avec la commande \ref{label} si besoin
%\end{center}
%\end{figure}

% Commande pour forcer toutes les images � �tre mises ici
%\FloatBarrier

% Appel � la fiche pour chaque trou avec la commande
%\input{Zones/PV/PV1}

% Page blanche si besoin
\cleardoublepage


 
% ==================================================================
% ANNEXES
\appendix
%\include{annexe} 
% Page Blanche si besoin
\clearpage

\newlength\mylength
\newcolumntype{C}[1]{>{\centering\arraybackslash}p{#1}}

{\scriptsize
\begin{center}
\setlength{\extrarowheight}{2pt}
\setlongtables

\begin{longtable} {|m{1cm}|m{1cm}|m{1cm} |m{1.8cm} |m{1.8cm} |m{1cm} |m{1.8cm} |m{1.8cm}|C{1cm}|}
\hline \hline Nom & Prof. (m) & Dev. (m) & X UTM (m) & Y UTM (m) & Altitude & Latitude (�) & Longitude (�) & Plaquette \\ \hline \hline \endhead
 \hline \hline \endfoot

	\rowcolor{lightgray} 
	A1 & -10 & 10 & XXXX & XXXX & 1929 & 46.00000 & 06.00000 & X \\\hline

	A1 & -10 & 10 & XXXX & XXXX & 1929 & 46.00000 & 06.00000 & X   \\\hline

	\rowcolor{lightgray} 
	A1 & -10 & 10 & XXXX & XXXX & 1929 & 46.00000 & 06.00000 & X \\\hline

	A1 & -10 & 10 & XXXX & XXXX & 1929 & 46.00000 & 06.00000 & X \\\hline

\end{longtable}  
\end{center}
}

% Page Blanche si besoin
%\cleardoublepage

 
% ==================================================================
% COLOPHON
%\colophon{Ce document a été préparé à l'aide de l'éditeur de texte GNU
%  Emacs et du logiciel de composition typographique \LaTeXe
%  Rajouter comment il est produit ???? et par qui ????.}
 
\end{document}		% Il faut toujours cette ligne pour finir un document
%%% Local Variables:
%%% mode: latex
%%% TeX-master: t
%%% End:
