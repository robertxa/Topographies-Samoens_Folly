% Page Blanche si besoin
\clearpage

{\scriptsize	% On change la casse pour du tout petit
\begin{center}	% On centre le tableau
\setlength{\extrarowheight}{2pt}
\setlongtables

\begin{longtable} {|m{1cm} |m{0.8cm} |m{0.8cm} |m{1.8cm} |m{7cm} |m{3cm}|}	% On ouvre le tableau et on d�finit les colonnes, leur taille et leur lignes verticales
\hline \hline textbf{Nom} & textbf{Prof. (m)} & textbf{Dev. (m)} & textbf{Date} & textbf{Remarques} & textbf{R�f�rences} \\ \hline \hline \endhead	% On d�finit le header
 \hline \hline \endfoot
% Dans un tableau, on s�pare chaque colonne avec &, et on revient � la ligne avec \\. \hline, c'est pour g�n�rer une s�paration horizontale sous forme de ligne
	\rowcolor{lightgray} % Ca permet de griser la ligne en cours
	BA1 & -1 & 1 & ??? & Puits �rod� dans la falaise des Barmes & ??? \\\hline

	BA2 & 2 & 10 & 30/05/09 & Galerie � d�sober & E.V. 67   \\\hline

	\rowcolor{lightgray} 
	CP1 & -60 & 80 & 01/06/74 & Glaci�re \/ Courant d'air � travers blocs & E.V. 16, 33, 34, 42*, 44, 47, 51, 53, 64 \\\hline

	CP2 & -51 & 51 & 05/08/74 & Courant d'air � travers �boulis & E.V. 31, 36, 58* \\\hline

\end{longtable}  
\end{center}
} % Fermer le bloc scriptsize

% Page Blanche si besoin
%\cleardoublepage
