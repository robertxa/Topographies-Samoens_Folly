%\newpage

%%%%%%%%%%%%%%%%%%%%%%%%%%%%%%%%%%%%%%%%%%%%%%%%%%%%%%%%%%%%%%%%%%%%%%%%%%%%%%%%%%%%%%%%%%%%%%%%%%%%%%
% D�finition des variables
\trou{CP2} % Nom(s) du trou

\systeme{Combe aux Puaires}	% Syst�me auquel il appartient

\developpement{51} % sans unit�
\profondeur{-51} % sans unit�

\lat{46,116134} % sans unit�
\longi{6,797549} % sans unit�
\altitude{} % sans unit�
\XUTM{329818} % sans unit�
\YUTM{5109309} % sans unit�

\situation{
entr�e de la combe aux puaires. (en bas des "grand trous")
}

\decouverte{
05/08/74
}

\histoire{
Explor� le 05/08/74 par P. Cahinght et Serge, Topo effectu�e (?). revu par Christian Rigaldie et Phillippe Lavabre en aout 77. \\
Revu le 12/07/00 par P. Cahingt et par S. Lips.
}

\descrip{
Puits de 51 m�tres dans une diaclase importante, il a une forme ovale caract�ristique. Le courant d'air se perd dans la salle terminale (4X10 m) � travers un �boulis instable et infranchissable.
}

\rem{
termin�, courant d'air
}

\biblio{
E.V. 31, 36, 58*
}

%%%%%%%%%%%%%%%%%%%%%%%%%%%%%%%%%%%%%%%%%%%%%%%%%%%%%%%%%%%%%%%%%%%%%%%%%%%%%%%%%%%%%%%%%%%%%%%%%%%%%%
% Construction la fiche

% rajout des figures (jpg ou png)
%\begin{wrapfigure}[20]{r}{6cm}
%	\begin{center}
%		\fbox{\includegraphics[width=1\linewidth]{Zones/CP/topos/CP02.png}}
%	\end{center}
%\end{wrapfigure}
%\fichetrou

\topopt{Zones/CP/topos/CP02.png}
\fichetroupt

% impose l'impression des images ici
\FloatBarrier

%\newpage
%\cleardoublepage
