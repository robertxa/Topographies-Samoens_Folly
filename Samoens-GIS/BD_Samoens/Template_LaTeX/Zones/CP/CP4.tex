%\newpage

%%%%%%%%%%%%%%%%%%%%%%%%%%%%%%%%%%%%%%%%%%%%%%%%%%%%%%%%%%%%%%%%%%%%%%%%%%%%%%%%%%%%%%%%%%%%%%%%%%%%%%
% D�finition des variables
\trou{CP4} % Nom(s) du trou

\systeme{Combe aux Puaires}	% Syst�me auquel il appartient

\developpement{22} % sans unit�
\profondeur{-22} % sans unit�

\lat{46,118052} % sans unit�
\longi{6,799403} % sans unit�
\altitude{2197} % sans unit�
\XUTM{329967} % sans unit�
\YUTM{5109518} % sans unit�

\situation{
sur les lapiaz bordant la combe aux puaires, sur la m�me faille que le CP05 et le CP06.
}

\decouverte{
07/08/74
}

\histoire{
explor� le 07/08/74 par Rigaldie C.
}

\descrip{
puits de 22 m sur faille obstru� par �boulis et assez �troit. Sur le cot�, � trois m�tres du fond, la faille se prolonge lat�ralement mais reste trop �troite.
}

\rem{}

\biblio{
E.V. 31, 36*
}

%%%%%%%%%%%%%%%%%%%%%%%%%%%%%%%%%%%%%%%%%%%%%%%%%%%%%%%%%%%%%%%%%%%%%%%%%%%%%%%%%%%%%%%%%%%%%%%%%%%%%%
% Construction la fiche

\topopt{Zones/CP/topos/CP4.png}
\fichetroupt

% impose l'impression des images ici
\FloatBarrier

%\newpage
%\cleardoublepage
