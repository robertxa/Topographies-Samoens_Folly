% Si nous voulons que ce soit sur une nouvelle page
%%% A commenter si la fiche est pour 1/2 page (PETIT TROU) %%%
\newpage

%%%%%%%%%%%%%%%%%%%%%%%%%%%%%%%%%%%%%%%%%%%%%%%%%%%%%%%%%%%%%%%%%%%%%%%%%%%%%%%%%%%%%%%%%%%%%%%%%%%%%%
% D�finition des variables
% Si une variable n'est pas d�finie, il faut la laisser vide, i.e. \trou{}

\trou{NOM_TROU} % Nom(s) du trou

\systeme{NOM_SYSEME}	% Syst�me auquel il appartient

\developpement{DEVELOPPEMENT} % sans unit�
\profondeur{PROFONDEUR} % sans unit�

\lat{LATITUDE} % sans unit�
\longi{LONGITUDE} % sans unit�
\altitude{ALTITUDE} % sans unit�
\XUTM{X_UTM} % sans unit�
\YUTM{Y_UTM} % sans unit�

\situation{
SITUATION DE LA CAVITE
}

\decouverte{
DATE DE DECOUVERTE
}

\histoire{
HISTORIQUE. \\
POUR LES RETOURS A LA LIGNE, IL FAUT LES REPERER DANS LA BD ORIGINALE, ET LES REMPLACER PAR \\
NE PAS RAJOUTER DE RETOUR A LA LIGNE POUR LE DERNIER PARAGRAPHE.
}

\descrip{
DESCRIPTION \\
POUR LES RETOURS A LA LIGNE, IL FAUT LES REPERER DANS LA BD ORIGINALE, ET LES REMPLACER PAR \\
NE PAS RAJOUTER DE RETOUR A LA LIGNE POUR LE DERNIER PARAGRAPHE.
}

\rem{
REMARQUES \\
POUR LES RETOURS A LA LIGNE, IL FAUT LES REPERER DANS LA BD ORIGINALE, ET LES REMPLACER PAR \\
NE PAS RAJOUTER DE RETOUR A LA LIGNE POUR LE DERNIER PARAGRAPHE.
}

\biblio{
BIBLIOGRAPHIE
}

%%%%%%%%%%%%%%%%%%%%%%%%%%%%%%%%%%%%%%%%%%%%%%%%%%%%%%%%%%%%%%%%%%%%%%%%%%%%%%%%%%%%%%%%%%%%%%%%%%%%%%
% Construction la fiche

%%% Dans le cas d'un GROS TROU (plus d'une 1/2 page) %%% :
	\fichetrougd

	% rajout des figures (jpg ou png), le faire pour chaque figure
	\begin{figure}[ht]
		\begin{center}
			\includegraphics[width=0.95\linewidth]{CHEMIN_VERS_MON_IMAGE.png}	% Ici, l'image est resiz�e � 0.95% de la largeur de la ligne
		\end{center}
	\end{figure}
	% impose l'impression des images ici
	\FloatBarrier

%%% Dans le cas d'un PETIT TROU (moins d'une 1/2 page) %%% :	
	% Chemin de la figure
	% La figure doit �tre au max de 5 cm de large pour 9 cm (?) de haut, sinon la fiche sera sur la page d'apr�s. 
	% Il faut faire attention � la largeur, mais aussi � la hauteur de l'image. 
	% Il faut peut �tre faire un algorithme de conversion des images vers pour avoir une hauteur et/ou une largeur inf�rieur � la limite max.
	\topopt{CHEMIN_VERS_MON_IMAGE.png}
	\fichetroupt
	% impose l'impression des images ici
	\FloatBarrier

% Si besoin d'une page blanche
%\clearpage
%\cleardoublepage
